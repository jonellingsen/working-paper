\section{Conclusion}\label{conclusion}

In this paper, I have shown how one type of big data, specifically search value indices from Google Trends, can provide valuable signals for nowcasting Norwegian retail sales and unemployment rate. By using least angle regression techniques in a dynamic way, I am able to pick the top predictors to be included in the nowcasting model, from a large set of SVIs I classify as potential predictors. The main gain from using SVIs for nowcasting is due to their timeliness. While retail sales and the unemployment rate are both published with approximately 1 month lag, the SVIs are available and updated each day throughout the current month. I show that some models that only includes SVIs related to unemployment as predictors, and no autoregressive terms, perform equally good at nowcasting the unemployment rate as an AR(1). Further, I find that the models that include the SVIs as predictors, in some cases perform substantially better than an AR(1) model during large shocks to the target variable, e.g. the financial crisis. Hence, it might be the case that the largest potential of the SVIs lies in their ability to nowcast sudden shocks, rather than underlying trends, better than AR models. Natural extensions to this paper would be to create more complex methods for aggregating single queries into broader factors related to the spesific target variables, and investigate the nowcasting performance of the SVIs throughout the month in order to exploit their timeliness.