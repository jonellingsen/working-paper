\section{Discussion}\label{discussion}

I have presented mixed results. On the one hand, my findings indicate that, even without any complex aggregation method, the SVIs can provide valuable information for nowcasting the unemployment rate. On the other hand, on average, I do not find particularly positive results for retail sales, even though some periods are more promising than others. Further, I show that the performance of the models that include SVIs as predictors are volatile over time, and hence non-trivial to interpret. Below I will briefly discuss what I perceive as the main challenges related to using SVIs to predict economic aggregates.

Firstly, it does not exist any clear theory of what type of queries that are relevant to investigate in the first place, and so the first step of collecting SVIs becomes somewhat subjective. 

Secondly, like every time series, the SVIs can be decomposed into different signals. In this paper, I try to identify some of the relevant sentiment-revealing categories and/or queries, that may work as proxies for the public interest in retail sales and unemployment, respectively. However, the SVIs contain a lot more signals than those related to the target variables that I am looking for. For example, when a person enters a query into Google that is related to retail sales in any form, this may not indicate that this person actually intends to do shopping. Maybe he seeks that information for a whole other reason than the intention to make a purchase, maybe even by plain curiosity. It seems plausible that this constitutes the main challenge we face when we use SVIs to nowcast economic aggregates. However, there is no easy solution to this problem. One starting point for adressing this challenge is to look at the aggregation of the SVIs, both in terms of the level and in terms of the composition of the aggregates. Intuitively, some combination of disaggregated and correlated SVIs may average out the noisy signals, and more accurately capture a common underlying trend. Potentially, the predefined categories from Google Trends could inhabit this property, but the results indicate that it may be more complex. As mentioned in Section \ref{introduction}, an alternative approach that might enhance the predictors, is to use methods like principal components, to extract a few common unobserved \textit{components/factors} that explain a large share of the variation in some set of correlated SVIs, see \textcite{vosen2012}. However, even though this might decrease the noise in the data, it does not necessarily capture the signals that are relevant for prediction, as the method of principal components does not take into account the target variables. Another more complex possibility would be to use dynamic factor models, that take into account the target variable when extracting the components, see \textcite{forni2000}.

Thirdly, as I emphasize in Section \ref{crisis}, inspired by \textcite{choi2012}, it might be that the strength of SVIs, does not lie in their ability to predict macroeconomic variables in ''normal'' times, but in their ability to predict turning-points better than autoregressive models, especially due to their timeliness. As I have shown, this might be the case also in the setting of this study. However, this should be looked deeper into, by investigating more periods subject to shocks to find out whether this truly is a significant property of the SVIs. In order to do so, a longer sample with more shocks is of course preferred. If the findings in this paper reflects a deeper property of the data, the SVIs could have significant impact for policy makers as an early indicator for current shocks to the economy.

Forthly, the SVIs are probably biased relative to the public interest. Ideally, the SVIs should represent the public interest in a particular subject or category. However, there are reasons to question the bias of the sample. For example, it is natural to assume that the young part of the population is overrepresented relative to the old part, due to technological challenges.

Fifthly, a central question in prediction is the lead/lag structure between the target variable and the predictors. I have considered a contemporaneous relationship, where I have allowed a lag of the SVIs to be included in addition. It might be that the lead/lag relationships are more complex. Several of the earlier studies in this field have used weekly data from Google Trends, and aggregated these SVIs into monthly frequency in different ways, see e.g. \textcite{anvik2010}, \textcite{choi2012} and \textcite{boe2011}. Since then, Google have made changes in the Google Trends API such that data from the whole sample is only reported on a monthly frequency. However, with some work, it is possible to obtain SVIs on higher frequency by combining different data sets, and this might be a valuable approach.

Finally, I want to stress that this paper only looks at linear models. Further work might look into the possibility that some of the relationsships between the target variables and the SVIs are non-linear. For instance, instead of restricting the model to be linear in each period, one could allow non-linearities in the estimation, and maybe capture the historical relationsship more accurate. However, non-linear models are of course more complex, and it might also be that introducing non-linearity increases the noise in the estimates and hence decrease the prediction accuracy.