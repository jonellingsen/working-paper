\begin{abstract}
Nowcasting economic aggregates with the use of timely data is an important challenge faced by central banks, and forecasting in general. I address this challenge by using timely search value indices (SVIs) from Google Trends to nowcast retail sales and the unemployment rate in Norway. By using least angle regression techniques to dynamically rank the top predictors from a large set of SVIs, I perform out-of-sample nowcasts where the nowcasting models include the top ranked SVIs. My findings show that all the nowcasting models perform far better than a random walk, and some of the models that only include SVIs as predictors perform equally good as an AR(1) in nowcasting the unemployment rate. However, my findings suggest that, on average, none of the SVIs provide any valuable information complementary to simple autoregressive models, like an AR(1). Nevertheless, due to their timeliness, the results indicate that SVIs may be valuable for nowcasting. Interestingly, when I only focus on the financial crisis period, where the Norwegian economy was hit by an unexpected large shock, some of the SVI models outperform an AR(1) by a substantial amount in nowcasting the unemployment rate. This finding suggest that the SVIs property as an early indicator detecting turning points, should be investigated further.
\end{abstract}