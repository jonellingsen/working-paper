\section{Introduction}\label{introduction}

Predicting key economic aggregates is an important task for policy makers, including central banks. The lack of macroeconomic variables measured in real-time has led economists to search for other types of data than the standard data from the national accounts, in order to assess the current economic fluctuations. A few examples are surveys, interest rate spreads and other types of high frequent financial data. In the litterature, this is known as \textit{nowcasting}. The basic principle of nowcasting is to use data, that are published earlier (more timely) and possibly also at a higher frequency than the target variable, as an early indicator before the official statistics are published, see \textcite{banbura2013}. Nowcasting the economy by using timely information sets, may improve policy decisions that, by nature, have to be made in real-time.
 
Nowadays, due to technological improvements, we face many new types of timely data. A particularly interesting trend is the evolution of so called \textit{Big Data}, see Section \ref{litterature}. From a nowcasting perspective, these new sources of information have the potential to provide more accurate assessments of economic fluctuations. However, as the name indicates, these data are BIG. The implication is that, in order to avoid bringing a lot of noise into the predictive models, we need selection methods to help us make sense of the large volume of data available.

A special type of Big Data is internet search data. Through their online service, Google Trends\footnote{See \url{www.google.com/trends}.}, Google publish disaggregated, near-real-time and high frequent data on internet search behavior. I refer to these series as search value indices (SVIs). According to \textcite{choi2012}, SVIs are often correlated with economic indicators. Due to their timeliness and the large volume of data available, SVIs may provide valuable information used to nowcast economic fluctuations. There are reasons to believe that there is a widespread use of Google's search engine in Norway. According to \textcite{ssb2015} and \textcite{ssb2016}, a large share of the households in Norway exploit the internet to obtain information. In 2016, 64 pct. of the respondents ordered goods online the last 12 months from a Norwegian producer. 86 pct. used the Internet for finding information about goods and services. Also, in 2015, 28 pct. of the respondents used the Internet to look for a job or sending a job application. It is plausible that many of these used search engines to find that information. According to \textcite{statcounter2016}, Google has approximately 90 pct. of the market share on search engines in Norway, making it a representative source for search activity.

For Norway, few studies have used SVIs to nowcast the economy\footnote{The only study, that I know of, is \textcite{anvik2010} who used SVIs to nowcast unemployment in Norway. However, this study had a very short sample compared to what is available today.}. I use monthly SVIs from Google Trends back to 2004, to nowcast two target variables - retail sales and the unemployment rate. I choose these target variables because they are reported on a monthly frequency and closely monitored by market analysts and policy makers. My hypothesis is the following: to the extent that people use Google as a source of information related to choices they are about to make, the development in the Google SVIs may reveal intentions driving economic events before they are captured by the official statistics. Hence, the SVIs work as proxies for the public interest in factors related to retail sales and unemployment.

As noted in \textcite{da2015}, the key to build an accurate predictive model with SVIs, is the identification of relevant sentiment-revealing search terms. In this paper, I do the following higher order steps. First, I collect timely data on SVIs for Norway from Google Trends. I download SVIs that I, subjectively believe to be potential predictors for the target variables. In particular, I use SVIs at different levels of aggregation - single queries and aggregated categories defined by Google - in separate analysis. This gives me a high dimensional data set, a typical characteristic of Big Data, consisting of more than 200 SVIs in total. Second, in order to build a nowcasting model that can give accurate out-of-sample predictions, I use the TS-LARS algorithm, developed by \textcite{gelper2008}, an algorithm that uses \textit{least angle regression}, see \textcite{efron2004}, to fit linear regression models to high-dimensional data. The TS-LARS ranks the SVIs, according to predictive power, and gives me a subset of the top ranked SVIs to include in the final nowcasting model, according to an information criterion.

In order to evaluate the performance of the nowcasting models, I divide the time series into two samples: a \textit{training} sample and a \textit{test} sample. I use the training sample to fit the models with the TS-LARS algorithm, and the test sample to evaluate their out-of-sample performance. The estimation is done both with an expanding and a rolling window. I compare the perfomance of the nowcasts from the SVI models to the performance of two simple benchmark models - an autoregressive model of order 1 (AR(1)), following \textcite{choi2012}, and a random walk.

I have three main findings. First, single query SVIs tend to be highly unstable over time, especially in the beginning of the sample period, from 2004 - 2006. This finding, as well as a desire to capture broader trends, suggests that the SVIs should be aggregated into broader measures, to capture the common underlying signals. Second, all the nowcasting models that include the top ranked SVIs as predictors perform far better than a random walk. Further, I find that the models that include SVIs, and not any autoregressive terms,  as predictors, in some cases, on average, perform equally good as the AR(1) in nowcasting the unemployment rate. However, I find that the SVIs, on average, do not provide any valuable information complementary to the AR(1). I stress that this paper uses end-of-month data. Hence, the advantage of using SVIs for nowcasting during the current month, before the data on the target variable for the previous month are released, is not quantified. Since some of the SVIs are equally good predictors as the lag of the target variable, and available earlier, using them for nowcasting throughout the current month might be valuable. Third, inspired by \textcite{choi2012}, I find that some of the nowcasting models that included SVIs as predictors outperformed the AR(1) during the financial crisis in 2008/2009. Hence, it might be that the largest potential in SVIs is related to their ability to nowcast sudden fluctuations more accurately than simple autoregressive models. 

My findings suggest that, for Norwegian data, further work is still to be done, in order to extract the valuable signals from the large set of SVIs available. An interesting extension to my study might be to use statistical methods to aggregate the query SVIs into categories based on their common variation, e.g. by using principal components or dynamic factor models. Further, it would be interesting to investigate the nowcasting performance of the SVIs throughout the month, in order to exploit their timeliness.

I have used the software R, see \textcite{r}, to perform all the analysis in the thesis. The codes are available upon request. The rest of this paper is organized as follows. Section \ref{litterature} describes Big Data, nowcasting and the litterature on the use of SVIs for prediction. Section \ref{data} describes the data. Section \ref{variable_selection} describes the variable selection methodology. Section \ref{out_of_sample} describes the out-of-sample exercise. Section \ref{empirical_results} presents the empirical results. Section \ref{discussion} provides a discussion of my approach to the problem, and possible future extensions to the paper. Section \ref{conclusion} concludes. The Appendix at the end provide additional information.